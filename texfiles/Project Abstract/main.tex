\documentclass{article}
\usepackage[utf8]{inputenc}
\usepackage{authblk}
\title{Facial Expression Recognition using Compact Embeddings}

\author{Josyula V N Taraka Abhishek \\
        \and 
        M V Sai Kiran  \\
        \and 
        Bhargav Chirumamilla \\
        \and
        Cheedrala Jaswanth \\
}
\date{September 2022}

\begin{document}

\maketitle

\section{Abstract}
Automatic facial expression recognition is a computer vision activity where we classify a photo into discrete expressions or emotions. However expressions do not always classify into
a specific class, for example there could be shy smile to laughter. So, we try to classify the image into a range of expressions using embeddings.
\\

\textbf{Embeddings in Machine Learning:}
An embedding is a low dimensional capture of high dimensional data. We use embedding techniques like Fourier descriptors to cpature face structure and classify them.
We plan to capture enough information about the picture to solve the problem at hand.

\section{Stages of project:}
\begin{enumerate}
    \item Understanding Problem Statement
    \item Defining the problem and scope of problem
    \item Collecting data
    \item Label and Organise data
    \item Pre-Process data
    \item Select model
    \item Train model
    \item Calculate error
    \item Take the best model
\end{enumerate}
\end{document}
